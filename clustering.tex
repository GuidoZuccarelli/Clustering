\documentclass[a4paper,10pt]{article}
\usepackage[utf8]{inputenc}
\setcounter{secnumdepth}{5}
\setcounter{tocdepth}{5}
%opening
\title{Clustering para la clasificación de ítems representados por la web semántica}
\author{Guido Zuccarelli}
\usepackage{graphicx}
\begin{document}

\maketitle

\begin{abstract}
Este trabajo mostrará la metodología, implementación y resultados de un caso en el cuál, partiendo de un dataset que contiene datos 
extraídos de la web semántica, se realizó un proceso de clasificación utilizando algún algoritmo de clustering de minería de datos 
para lograr que los ítems del dataset que representan productos, posean un atributo label, que indique de qué tipo de producto se 
trata. \\
Este proceso se convierte en un paso de integración en un trabajo de desarrollo de una aplicación que utilice datos de la web semántica, 
ya que se agruparán ítems mediante los clusters a los cuales fueron asignados.
\end{abstract}
\newpage
\tableofcontents

\include{introduccion}

\section{Problema planteado}

Se dispone de un dataset semántico que contiene 16661 recursos de tipo Producto (clase definida bajo la ontología schema.org). 
Dicho tipo, no es la mejor representación que puede obtener el ítem, dado que un Producto es un tipo muy abarcativo y puede representar 
también, múltiples otros tipos de clase. Como por ejemplo, un producto puede también ser un utensillo, o música. \\
\\
La ontología schema, define una taxonomía de 946 clases, y esos 16661 ítems, deberían estar mejor distribuidos bajo la mismo, que agrupados 
todos juntos en la clase Producto. \\
Para esto se dispone de información representativa de cada ítem, que ayudará a describir al mismo e intentar establecer un nuevo tipo que 
defina al mismo.

\section{Objetivos}

En base al caso de estudio planteado en la sección  anterior, mediante la implementación de algún algoritmo de clustering de minería de datos 
se definen los siguientes objetivos:

\begin{enumerate}

 \item Agrupar los ítems, que representan el mismo dominio de información. Esto es, reconocer los recursos que tienen el mismo tipo,
 como por ejemplo, agrupar todos los ítems de tipo película
 
 \item Etiquetar aquellos grupos encontrados de acuerdo al tipo de información que representan.
 
 \item Reconocer a qué clase de la ontología schema.org pertenecen aquellos grupos encontrados, si es que existe alguna.
 
 \item Verificar la performance de los clusterings realizados en base a métricas ya definidas.
 
 \item Evaluar las distintas posibles configuraciones del algoritmo de clustering elegido.
 
 \item Intentar establecer una relación entre la performance obtenida para un agrupamiento encontrado, y el éxito obtenido en corresponder 
 los grupos con clases de la taxonomía definida por schema.org
 
\end{enumerate}


\include{estrategia}

\include{resultados}

\include{conclusiones}
\end{document}
