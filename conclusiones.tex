\section{Conclusiones}

En lineas generales los resultados obtenidos fueron muy satisfactorios, se pudieron clasificar el 53\% de los productos con tan solo un 
4\% de margen de error aproximadamente. \\
\\
Hay un marcado progreso en cuanto a la eficiencia de los los modelos de clustering a medida que la cantidad de centroides aumenta. Tanto 
considerando el análisis algorítmico como el análisis experto. También considerando la utilidad del modelo resultante para lograr llevar a cabo el objetivo
de etiquetar dichos ítems, sumado a la disminución de etiquetas incorrectas.\\
Esta situación tiene fundamento en que al ser textos redactados por distintos usuarios la fuente de información con la que se trabaja, provoca que 
los individuos (ítems) estén muy dispersos entre sí. Esto parece provocar que a mayor división entre ellos, mejores resultados se reflejan, considerando los 
parámetros mencionados. De manera que si se utiliza un número mayor o igual de clusters que de ítems, quedando así un ítem por cluster, 
el resultado tanto del índice silouhette como de la efectividad de los clusters sería perfecto.\\
Sin embargo, aumentar considerablemente el número de centroides llevaría a claros problemas, como la cantidad de esfuerzo humano que se requeriría 
para analizar los clusters, o el tiempo de ejecución de los algoritmos. Que es justamente lo que se quiere reducir al utilizar clustering.\\
Por lo tanto el criterio de selección de la cantidad de centroides, debería hacerse considerando la relación entre la mejora que se podría obtener 
aumentando los centroides como la cantidad de esfuerzo horas hombres disponibles.\\

