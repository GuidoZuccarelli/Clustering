\section{Introducción}

\subsection{Dataset semántico}

Un dataset semántico está compuesto por un conjunto de documentos extraídos de la web, que a su vez poseen un conjunto de recursos
clasificados según su representación en el mundo real. Estos recursos pueden representar objetos tanto físicos como abstractos como por ejemplo: Autos, Libros, Ideas, 
Servicios, Plantas, etc.
Este dataset se almacena en una base de datos, la cuál tiene una forma especial para acceder a los datos. Mediante consultas SPARQL, o 
mediante la utilización de frameworks como Jena.
Dichos recursos pueden relacionarse entre sí mediante lo que se denomina propiedades, como por ejemplo un Recurso libro, puede estar 
relacionado, con un recurso Review, mediante la propiedad ``review''. De manera que un documento semántico puede representarse en forma de grafo, 
donde cada nodo es un recurso, y cada arista es una propiedad.


\subsection{Ontologías}

Es la representación abstracta de la taxonomía de los ítems, bajo la cual se encuentran definidos todos los recursos de la web semántica.
Está formada por un conjunto de clases y relaciones entre ellas, de manera que se puede definir un ítem como individuo de una o más clases 
de cualquier ontología. Cada clase y relación está definida bajo un namespace que la identifica unívocamente.
Existen, muchas distintas ontologías en la web que modelan diferentes dominios. 
La clase de un ítem, también puede ser llamada tipo. Y funciona de manera similar a la programación orientada a objetos. 

\subsection{Item}

Llamaré ítem, a los recursos de clase ``Producto'' del dataset que estén relacionados con al menos un recurso Review. Estos recursos pueden reprensentar 
múltiples tipos de producto, como por ejemplo: Películas, Ropa, Hardware, etc, los cuales se intentará identificar mediante la clusterización.
Estos ítems tendrán información representativa, establecida por propiedades y otros recursos.

\subsection{Review}

Es un recurso de clase ``Review'' que se relaciona con un recurso Ítem mediante la propiedad ``review''. Representa la evaluación de dicho ítem 
de parte de un usuario y está compuesto principalmente por un puntaje numérico y una evaluación textual. \\

\subsection{K-Medias}

Es un algoritmo de agrupamiento que utiliza aprendizaje no supervisado para resolver una clusterización, en K número de clusters (prefijado).\\
Utiliza K puntos aleatoria o arbitrariamente dispersos en el mapa, que serán llamados centroides. El algoritmo entonces, en varias etapas (también prefijadas) 
intenta aproximar esos puntos a los clusters, asignándoles los elementos más cercanos.
Se pueden utilizar varias medidas de distancias, entre las que se encuentran la distancia euclidiana cuadrada, o la distancia Itakura Saito.

\subsection{Modelo de clustering}

Es el resultado de aplicar un algoritmo de clustering, que esta compuesto por la ubicación de sus centroides. Dichos modelos pueden ser utilizados 
agrupar los datos, simplemente con asignarlos al centroide más cercano. 

\subsection{TF-IDF}

Es una de las más populares medidas que se usa para representar el valor numérico de la información de un documento de texto estructurado en una lista de palabras en consideración a un conjunto de documentos.
Esta medida pondera cada palabra de un texto, otorgándole un puntaje que depende de cuántas veces dicha palabra se encuentra en un documento, y cuán frecuente 
es esa palabra en otros documentos.

\subsection{Silohouette Coefficient}

Es una medida que le otorga un valor numérico entre -1 y 1 a la correctitud de la ubicación de un ítem en un cluster, para luego, otorgar un mismo 
rango de puntaje a un modelo de clustering, promediando los valores del conjunto de ítems aplicados a dicho modelo.\\
Esta medida considera la distancia del ítem con respecto al conjunto de ítems de su mismo cluster, y luego al resto de los ítems. De manera que,
cuanto menor distancia haya con respecto a los de su mismo cluster y mayor distancia con respecto a ítems de clusters foráneos, mejor puntaje obtendrá.

\subsection{SPARQL}

Es un lenguaje de consultas que se utiliza para acceder y modificar información de un dataset semántico. Para esto se utiliza un motor específico que depende 
de cómo esté almacenado el dataset semántico. Suele ser bastante flexible, pero no admite demasiada manipulación de los datos resultantes. 
Se puede combinar la información que retorna pero suele ser con queries muy complejas que además demandan mucho timepo de ejecución.